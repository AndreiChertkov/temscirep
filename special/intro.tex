Приводим абзац текста, с описанием текущего состояния дел по нашему направлению исследований, и формулируем целевую масштабную нерешенную научно-техническую проблему.

Соответственно основными целями проведения данной научно-исследовательской работы являются (обычно можно скопировать цель или цели из ТЗ):
\begin{enumerate}
	\item Первая цель работы;
	\item Вторая цель работы;
	\item Третья цель работы.
\end{enumerate}

Основанием для проведения работ является ТЗ на научно-исследовательскую работу по теме <<Спасение мира с использованием тензорных поездов>>, оформленное как приложение к Договору от 01.01.22 № 424242, заключенному между Министерством Добрых Дел и АНОО ВО <<Сколковский институт науки и технологий>>.
При этом, для успешного достижения поставленных целей на данном, первом, этапе работ по Проекту, было запланировано решение следующих задач (обычно можно скопировать задачи из ТЗ):
\begin{enumerate}
	\item Первая задача работы на этапе;
	\item Вторая задача работы на этапе;
	\item Третья задача работы на этапе;
	\item Четвертая задача работы на этапе;
	\item Пятая задача работы на этапе.
\end{enumerate}

Все поставленные задачи были выполнены нами в полном объеме, и далее в отчете мы описываем соответствующие основные полученные результаты.
В разделе~\ref{sec:demo} мы приводим разработанный \ldots.
В разделе \ldots.
Наконец, в Заключении формулируются основные выводы по проведенной на данном этапе работе и обсуждаются возможные пути ее дальнейшего развития.

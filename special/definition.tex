В настоящем отчете о научно-исследовательской работе применяются следующие обозначения и сокращения.

\vspace{0.4cm}

\newcommand{\abbr}[2]{\begin{flushleft}#1 – #2.\\\end{flushleft}}

\abbr{API}
  {программный интерфейс приложения (Application Programming Interface)}

\abbr{GUI}
  {графический интерфейс пользователя (Graphical User Interface)}

\abbr{ML}
  {машинное обучение (Machine Learning)}

\abbr{SVD}
  {сингулярное разложение (Singular Value Decomposition)}

\abbr{TT-разложение}
  {разложение тензорного поезда (Tensor Train decomposition)}

\abbr{TT-тензор}
  {тензор, представленный в форме разложения тензорного поезда}

\abbr{Вектор}
  {одномерный упорядоченный массив вещественных или комплексных чисел (является также одномерным тензором)}

\abbr{ИИ}
  {Искусственный Интеллект}

\abbr{ИНС}
  {Искусственная Нейронная Сеть}

\abbr{Матрица}
  {двумерный упорядоченный массив вещественных или комплексных чисел (является также двумерным тензором)}

\abbr{Отчет}
  {данный отчет о НИОКР}

\abbr{ПО}
  {Программное Обеспечение}

\abbr{ТЗ}
  {Техническое Задание}

\abbr{Тензор}
  {одномерный или многомерный упорядоченный массив вещественных или комплексных чисел}

\abbr{ЭО}
  {Экспериментальный Образец}

\abbr{$\matr{I}_{J} \in \set{R}^{J \times J}$}
  {диагональная единичная матрица размера $J \times J$}

\abbr{$\tens{G}[:,\,i,\,:]$}
  {матрица, полученная как срез трёхмерного тензора $\tens{G}$ при фиксированной второй размерности}

\abbr{$\tens{A} \circ \tens{B}$}
  {Адамарово (поэлементное) произведение тензоров $\tens{A}$ и $\tens{B}$}

\abbr{$\matr{A} \otimes \matr{B}$}
  {Кронекерово произведение матриц $\matr{A}$ и $\matr{B}$}

\abbr{$\matr{A}^T$}
  {матрица, транспонированная к $\matr{A}$}

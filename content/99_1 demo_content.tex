В процессе работы над отчетом могут (и должны) редактироваться следующие файлы:
\begin{enumerate}
  \item <<main.tex>> --
    основной файл проекта.
    При добавлении нового раздела отчета в папку <<content>>, ссылка на раздел должна быть добавлена в данный файл;

  \item <<content/>> --
    папка для хранения основных разделов отчета.
    Ссылки на разделы должны быть указаны в файле <<main.tex>>.
    В репозитории данная папка содержит раздел с рекомендациями по оформлению (<<content/99\_demo.tex>> и файлы с подразделами);

  \item <<images/>> --
    папка для хранения графических изображений.
    Отметим, что в latex-команде <<includegraphics>> достаточно указывать только имя файла (без указания папки).
    В репозитории данная папка содержит несколько примеров изображений;

  \item <<biblio.bib>> --
    файл с библиографией в bibtex-формате.
    Поля библиографической записи (англоязычные) автоматически приводятся к нижнему регистру.
    При необходимости сохранения заглавных букв, их нужно заключать в фигурные скобки;

  \item <<commands.tex>> --
    файл содержит различные полезные команды (выделение векторов, матриц, тензоров и~т.п.).
    В конце данного файла могут быть заданы собственные команды и сокращения;

  \item <<special/title.tex>> --
    файл содержит титульный лист отчета.
    См. также титульный лист для мегагранта в файле <<special/title\_mega.tex>> (переключение на мегагрант осуществляется посредством задания команды <<mega>> в файле <<main.tex>>);

  \item <<special/performers.tex>> --
    файл содержит список исполнителей отчета;

  \item <<special/abstract.tex>> --
    файл содержит реферат отчета;

  \item <<special/definition.tex>> --
    файл содержит перечень обозначений и сокращений;

  \item <<special/intro.tex>> --
    файл содержит введение к отчету;

  \item <<special/conclusion.tex>> --
    файл содержит выводы по отчету.
\end{enumerate}

Также проект содержит следующие файлы, редактирование которых не предполагается:
\begin{enumerate}
  \item <<README.md>> --
    файл с общим описанием репозитория в markdown формате;
  \item <<gost.bst>> --
    стандартный стилевой файл для оформления библиографии по ГОСТу;
  \item <<style.sty>> --
    наш набор стилей для оформления отчета в соответствии с ГОСТ.
\end{enumerate}

В данном разделе мы приводим основные рекомендации по оформлению отчета (в будущем они будут уточнены).
Отметим, что некоторые из рекомендаций могут оказаться нелогичными и / или неверными (если вы заметили подобную рекомендацию, то, пожалуйста, сообщите об этом разработчикам шаблона).

Если раздел содержит подразделы, то в разделе должно также обязательно присутствовать заключение (соответствующий подраздел может называться <<Выводы>> или <<Выводы по разделу>>), а также должно быть введение (вместо создания специального подраздела <<Введение>>, мы приводим для компактности вводный текст непосредственно в начале раздела до первого подраздела -- это не является грубым нарушением ГОСТа, однако при желании можно формально добавлять подразделы <<Введение>>).
См. в качестве примера раздел~\ref{sec:demo} и раздел~\ref{sec:demo_concl}.

В качестве названия разделов во многих случаях стоит указывать в точности названия пунктов плана работ из Соглашения по гранту или контракту (часто заказчики это требуют явно), чтобы проверяющий мог осуществить формальную проверку соответствия отчета пунктам официального ТЗ.
Во введении к разделу стоит указать общие мысли по теме, также можно описать, что будет содержаться в последующих подразделах (см. пример далее).

В разделе~\ref{sec:demo_content} мы детально опишем структуру данного проекта, приведем перечень редактируемых файлов в репозитории и их назначение.
Далее в разделе~\ref{sec:demo_examples} мы рассмотрим примеры оформления различных элементов отчета, таких как таблицы и графические изображения.
Затем в разделе~\ref{sec:demo_comments} мы сформулируем общие рекомендации по оформлению отчета о научно-исследовательской работе.

% Далее приводятся подразделы отчета из папки "content" в стиле:
% \subsect{файл из папки "content"}{метка для ref}
%   {отображаемое название подраздела}

\subsect{99_1 demo_content}{sec:demo_content}
    {Описание редактируемых файлов}

\subsect{99_2 demo_examples}{sec:demo_examples}
    {Примеры оформления различных элементов}

\subsect{99_3 demo_comments}{sec:demo_comments}
    {Общие комментарии}

\subsect{99_4 demo_concl}{sec:demo_concl}
    {Выводы по разделу}

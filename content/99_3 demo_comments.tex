\begin{enumerate}
  \item
    Списки стоит употреблять только типа <<enumerate>>, начинать пункты с заглавной буквы, в конце всех пунктов (кроме последнего) ставить точку с запятой;
  \item
    Кавычки стоит делать вот так <<привет>>;
  \item
    Абзацы текста должны содержать не менее трех предложений;
  \item
    Числа в тексте стоит оформлять как инлайновые формулы, например $42$;
  \item
    Ссылки стоит соединять с текстом тильдами для неразрывности, то есть не <<рисунок \ref{fig:tensor_train}>>, а <<рисунок~\ref{fig:tensor_train}>> или <<см. раздел~\ref{sec:demo_examples}>>;
  \item
    Стандартные сокращения стоит использовать с тильдами: <<и~т.д.>>, <<и~т.п.>>;
  \item
    Для именования разделов в папке <<content>> удобно использовать систему следующего типа: <<99\_3 demo\_comments>>, то есть последовательно указывается номер раздела, нижнее подчеркивание, номер подраздела, пробел, имя раздела, нижнее подчеркивание, имя подраздела.
    В данном случае в качестве метки на подраздел можно указать <<sec:demo\_comments>> (см. раздел~\ref{sec:demo_comments}).
    Подобное единообразие позволит быстро указывать верные метки в тексте, а также визуально различать разделы и подразделы в файловом менеджере.
    Отметим, что мы используем номер <<99>> для данного раздела, чтобы его всегда можно было включить в черновую версию отчета в качестве последнего раздела (для справки по оформлению);
  \item
    Предложения в latex файле удобно оформлять с новой строки -- это в ряде случаев позволяет быстрее локализовать ошибку, а также упрощает процесс комментирования ненужных предложений;
  \item
    Стоит не забывать использовать <<-->> вместо <<->>, когда это требуется;
  \item
    Стоит аккуратно оформлять библиографические записи. В качестве тегов стоит использовать (как и в системе Google Scholar) <<фамилияГОДслово>> (где <<слово>> -- это первое слово названия статьи; если в слове имеется дефис, то его следует убрать), например, <<oseledets2011tensortrain>> (см.~\cite{oseledets2011tensortrain}).
    Стоит использовать единый способ представления ФИО (например, <<Фамилия, И.>>).
    Иногда удобно располагать библиографические записи в файле <<biblio.bib>> в алфавитном порядке;
  \item
    Для <<строковых>> символов стоит использовать нижнее троеточие: <<$a, \, b, \, \ldots, \, z$>>, а для <<центрированных>> символов можно вот так: <<$a \times b \times \cdots \times z$>>;
  \item
    Стоит использовать единую систему абзацных отступов в latex коде внутри блоков (графические изображения, таблицы и~т.п.) для единообразия;
  \item
    Перед формулами ставим двоеточие, а после формулы точку или запятую;
  \item
    Подписи к рисункам и таблицам начинаем с большой буквы, в конце точку не ставим, в тексте ссылаемся без сокращений: <<см. рисунок~\ref{fig:tensor_train}>> или <<приводится в таблице~\ref{tbl:demo}>>;
  \item
    Хорошо если на последней странице раздела остается в конце мало пустого места;
  \item
    Букву <<ё>> употреблять в тексте не стоит.
\end{enumerate}

% \fat - bold font inside the formula or text:
\newcommand{\fat}[1]{\ifmmode\bm{#1}\else\textbf{#1}\fi}

% \set - linear/functional space (R, N, ...):
\newcommand{\set}[1]{\mathbb{#1}}

% \vect - vector:
\newcommand{\vect}[1]{\fat{#1}}

% \matr - matrix:
\newcommand{\matr}[1]{#1}

% \tens - tensor (or TT-tensor):
\newcommand{\tens}[1]{\mathcal{#1}}

% \func - function:
\newcommand{\func}[1]{\textsf{#1}}

% \vfunc - vector-function:
\newcommand{\vfunc}[1]{\textsf{\vect{#1}}}

% \ffunc - functional:
\newcommand{\ffunc}[1]{\widehat{#1}}

% \oper - operator:
\newcommand{\oper}[1]{\widehat{#1}}

% \trace - trace of the matrix:
\newcommand{\trace}[1]{\mathrm{Tr}\left( #1 \right)}

% \order - Operator O(...):
\newcommand{\order}[1]{\mathcal{O}\left( #1 \right)}

% \divergence - operator divergence(...):
\newcommand{\divergence}[1]{\mathrm{div}\left[ #1 \right]}

% \der - derivative d * / d *:
\newcommand{\der}[2]{\frac{d \, #1}{d \, #2}}

% \dert - time derivative d * / d t:
\newcommand{\dert}[1]{\der{#1}{t}}

% \derx - spatial derivative d * / d x:
\newcommand{\derx}[1]{\der{#1}{x}}

% \pder - partial derivative d * / d *:
\newcommand{\pder}[2]{\frac{\partial #1}{\partial #2}}

% \pdert - partial time derivative d * / d t:
\newcommand{\pdert}[1]{\pder{#1}{t}}

% \pderx - partial spatial derivative d * / d x:
\newcommand{\pderx}[1]{\pder{#1}{x}}

% \pderxi - partial spatial derivative for i-th component:
\newcommand{\pderxi}[2]{\pder{#1}{\vect{x}_{#2}}}

% \vectorize - vectorization (tensor to vector):
\newcommand{\vectorize}[1]{\fat{vec}\left( #1 \right)}

% \vectl - vector, represented as a list of its elements:
\newcommand{\vectl}[1]{\fat{\left[} #1 \fat{\right]}^{\top}}

% \iflatten - flatten index from multi-index:
\newcommand{\iflatten}[1]{\mathrm{ind}\left( #1 \right)}

% \imulti - multi-index from the flatten index:
\newcommand{\imulti}[1]{\fat{mind}\left( #1 \right)}

% \poi : point (dot) placeholder:
\newcommand{\poi}[0]{\, \cdot \,}

% \upd - update text content:
\newcommand{\upd}[1]{ {\color[HTML]{8b1d1d}#1} }

% \note - note (for debug):
\newcommand{\note}[1]{ {\color[HTML]{8b1d1d}/* #1 */} }
% \renewcommand{\note}[1]{} % Comment this for debug

% ... здесь можно добавить собственные команды ...
